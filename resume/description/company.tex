\resheading{项目经历}
  \begin{itemize}[leftmargin=*]
    \item
      \ressubsingleline{高速交通大数据分析}{路网中心}{2018.11 -- 2019.01}
      {\small
      \begin{itemize}
        \item 搭建hadoop分布式集群存储环境,对路网中心上传的\textbf{月数据量达500G}全国收费站数据进行存储管理
        \item 在交易数据记录中按属性进行聚类,检测出离群点,并将这部分离群点数据的信息完整杜进行评估,过滤出存在逃费的异常行为记录%
        \item 利用spark工具,对交易数据进行分析,生成不同省份的出行基本特征、区域运输特征以及通行费用基本特征的画像。 根据分析结果,给出合理的高速收费标准,并针对不同的费率模拟推演计算赢亏数量
      \end{itemize}
      }
%    \item
%      \ressubsingleline{论文发表}{Deep Learning based Energy-efficient Computational offloading in IOV}{2018.01 -- 2018.06}
%      {\small
%      \begin{itemize}
%        \item 参与完成论文的idea模型的构建
%        \item 参与完成论文的仿真实验部分,利用pythorch搭建CNN模型进行训练
%        \item 参与完成论文的Related work 和 Experiments 部分撰写
%        \item 投稿China Communication(已录用)
%      \end{itemize}
%      }
	\item
	  \ressubsingleline{高速路链数据拓扑分析}{Java/路网中心}{2018.11 -- 2019.01}
	  {\small
      \begin{itemize}
      	\item 对“路链"对象进行抽象封装,采用双向链表的数据结构
        \item 读取mapinfo文件生成路链拓扑结构,依据经纬度信息,将某省所有收费站映射到路链
        \item 依据路链拓扑信息,采用Dijkstra计算不同高速收费站间最短路链集合、距离及行驶时间
        \item 依据路链拓扑信息和站间流量,通过Map、Reduce计算得到每个路链上承载的货运量和通行量,并对这些路链进行聚类以标识高速关键路段
      \end{itemize}
      }
    \item
      \ressubsingleline{数据可视化网站搭建}{Python/后端}{2019.01 -- 2019.02}
      {\small
      \begin{itemize}
      	\item 基于高速交通大数据的分析结果完成数据可视化
        \item 后端采用Django框架,通过对数据分析结果数据库的增删查为前端提供数据请求接口
        \item 前端采用Echarts框架,对后台的数据加载渲染
      \end{itemize}
      }
  \end{itemize}