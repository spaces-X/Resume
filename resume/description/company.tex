\resheading{项目经历}
  \begin{itemize}[leftmargin=*]
    \item
      \ressubsingleline{高速交通大数据分析}{路网中心}{2018.06 -- 2018.09}
      {\small
      \begin{itemize}
        \item 搭建hadoop分布式集群存储环境,对路网中心上传的\textbf{月数据量达200G}全国收费站数据进行落地存储管理
        \item 在交易数据记录中按属性进行聚类,检测出离群点,并将这部分离群点数据的信息完整杜进行评估,过滤出存在逃费的异常行为记录%
        \item 利用spark等大数据工具,对交易数据进行分析,生成不同省份的出行基本特征、区域运输特征以及通行费用基本特征的画像
      \end{itemize}
      }
    \item
      \ressubsingleline{论文发表}{Deep Learning based Energy-efficient Computational offloading in IOV}{2018.01 -- 2018.06}
      {\small
      \begin{itemize}
        \item 参与完成论文的idea模型的构建
        \item 参与完成论文的仿真实验部分,利用pythorch搭建CNN模型进行训练
        \item 参与完成论文的Related work 和 Experiments 部分撰写
        \item 投稿China Communication(已录用)
      \end{itemize}
      }
    \item
      \ressubsingleline{数据可视化网站搭建}{Python/后端开发}{2017.10 -- 2018.01}
      {\small
      \begin{itemize}
      	\item 基于高速交通大数据的分析结果完成数据可视化
        \item 后端采用Django框架,通过对数据库的增删查为前端提供数据请求接口
        \item 前端采用Echarts框架,采用ajax对后台的数据对div进行异步加载渲染
      \end{itemize}
      }
  \end{itemize}