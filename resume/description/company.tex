\resheading{项目经历}
  \begin{itemize}[leftmargin=*]
    \item
      \ressubsingleline{高速交通大数据分析}{路网中心}{2018.11 -- 2019.01}
      {\small
      \begin{itemize}
        \item 搭建hadoop分布式集群存储环境,对路网中心上传的\textbf{月数据量达500G}全国收费站数据进行存储管理
        \item 在交易数据记录中按属性进行聚类,检测出离群点,并将这部分离群点数据的信息完整性进行评估,过滤出存在逃费的异常行为记录%
        \item 利用spark工具,对交易数据进行分析,生成不同省份的出行基本特征、区域运输特征以及通行费用基本特征的画像。 根据分析结果,给出合理的高速收费标准,并针对不同的费率模拟推演计算赢亏数量
        \item 依据路链拓扑信息,采用Dijkstra计算不同高速收费站间最短路链集合、距离及行驶时间,路段每日累计承担重量
      \end{itemize}
      }
  %  \item
  %    \ressubsingleline{论文发表}{Deep Learning based Energy-efficient Computational offloading in IOV}{2018.01 -- 2018.06}
  %    {\small
  %    \begin{itemize}
  %      \item 参与完成论文的idea模型的构建
  %      \item 参与完成论文的仿真实验部分,利用pythorch搭建CNN模型进行训练
  %      \item 参与完成论文的Related work 和 Experiments 部分撰写
  %      \item 投稿China Communication(已录用)
  %    \end{itemize}
  %    }
	% \item
	%   \ressubsingleline{高速路链数据拓扑分析}{Java/路网中心}{2018.11 -- 2019.01}
	%   {\small
  %     \begin{itemize}
  %     	\item 对“路链"对象进行抽象封装,采用双向链表的数据结构
  %       \item 读取mapinfo文件生成路链拓扑结构,依据经纬度信息,将某省所有收费站映射到路链
  %       \item 依据路链拓扑信息,采用Dijkstra计算不同高速收费站间最短路链集合、距离及行驶时间
  %       \item 依据路链拓扑信息和站间流量,通过Map、Reduce计算得到每个路链上承载的货运量和通行量,并对这些路链进行聚类以标识高速关键路段
  %     \end{itemize}
  %     }
    \item 
      \ressubsingleline{联通信令数据挖掘}{联通/安科院}{2019.03 -- 2019.10}
      {\small
      \begin{itemize}
        \item 基于联通的手机信令切片数据(80G/天),预处理对基站漂移和基站乒乓切换产生异常数据进行剔除
        \item 针对每个用户id,对个人每日的停留点进行提取,进而得到个人用户的出行链
        \item 对出行链中的停留点属性判断(职住),并对特定区域(大兴化工厂)工作人员的夜间活动人数跟踪,并与工业台账对比
        \item 对固定出行模式的人员进行识别(如: 外卖快递人员、出租车或网约车司机)
        \item 基于联通实时数据(5T/天)(Kafka消费)结合公交智能卡数据,对密集人群区域监控(南锣鼓巷、东直门枢纽)
      \end{itemize}

      }
    \item 
      \ressubsingleline{广告数据平台告警服务}{暑期实习/\textbf{字节跳动}}{2019.06 -- 2019.08}
      {\small
      \begin{itemize}
        \item 广告数据平台后台ABTest模块对实验组、对照组数据指标进行查询
        \item 采用孤立森林的方法,进行异常点检测并通过Lark消息对实验人进行报警
        \item 分析报警后台任务时间较长的原因,并将后台数据库由Mysql优化为Redis内存数据库
        \item 发送消息后响应用户(实验人)的反馈,将误判的实验id加入白名单
        \item 将原来每15分钟执行的任务,缩短到分钟级
      \end{itemize} 
      }
  \end{itemize}